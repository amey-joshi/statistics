\chapter{Introduction}\label{c1}
\section{Notation and definitions}\label{c1s1}
We will follow the contemporary mathematical style of using unadorned symbols for vectors and
matrices. The nature of a symbol will be determined by its definition and not appearance. For
example, $x$ could be a real number or a point in $\sor^n$ depending on the context.

If $f$ is a function of $n$ variables, $x_1, \ldots, x_n$ then its gradient is defined as
\begin{equation}\label{c1s1e1}
Df := \left(\frac{\partial f}{\partial x_1}, \ldots, \frac{\partial f}{\partial x_n}\right).
\end{equation}

\section{Problems}
\begin{enumerate}
\item In this problem, $w = (w_0, \ldots, w_M) \in \sor^{M+1}$, $x, y \in \sor$. Likewise, the $n$
observations $y_1, \ldots, y_n$ for each one of $x_1, \ldots, x_n$ are also real numbers. $y$
is modelled as a polynomial in $x$. That is,
\begin{equation}\label{c1pe1}
y = y(x, w) = \sum_{i=0}^Mw_ix^i.
\end{equation}
The model error is given by
\begin{equation}\label{c1pe2}
E(w) = \frac{1}{2}\sum_{j=1}^N \left(y(x_j, w) - t_j\right)^2,
\end{equation}
where $t_j$ is the true value of $y$ at $x_j$. The goal of the exercise is to choose $w$ such 
that for a given set of observations $(x_j, t_j)$, $E$ is minimal. Therefore, we find the
gradient of $E$,
\[
DE = \sum_{j=1}^N\left(y(x_j, w) - t_j\right)Dy.
\]
The gradient of $y$ with respect to $w$ is
\[
Dy = \sum_{i=0}^M x^i,
\]
so that
\[
DE = \sum_{j=1}^N\left(y(x_j, w) - t_j\right)\sum_{i=0}^M x_j^i.
\]
The condition for an extremum is $DE = 0$. The vector $DE$ is zero if and only if each one
of its components is zero.
\[
\sum_{j=1}^N\left(\sum_{k=0}^M w_k x_j^{k+i} - t_jx_j^i\right) = 0, \forall i = 0, \ldots, M
\]
We can simplify it as
\[
\sum_{k=0}^M \sum_{j=1}^N w_k x_j^{k+i} = \sum_{i=0}^M\sum_{j=1}^N t_jx_j^i.
\]
Let,
\begin{eqnarray*}
T_i &=& \sum_{j=1}^N t_jx_j^i \\
A_{ik} &=& \sum_{j=1}^N x_j^{k+i}
\end{eqnarray*}
so that the condition from an extremum becomes
\[
\sum_{k=0}^M A_{ik}w_k = T_i
\]

\item The regularized error function is
\begin{equation}\label{c1pe3}
E(w) = \frac{1}{2}\sum_{j=1}^N \left(y(x_j, w) - t_j\right)^2 + \frac{\lambda}{2}\norm{w}^2.
\end{equation}
Its gradient is
\[
DE = \sum_{j=1}^N\left(y(x_j, w) - t_j\right)Dy + \lambda w,
\]
where we have used the fact that 
\[
D\norm{w}^2 = \left(\frac{\partial}{\partial w_0}\sum_{i=0}^M w_i^2, \ldots, \frac{\partial}{\partial w_M}\sum_{i=0}^M w_i^2\right) = 2w.
\]
The vector $DE = 0$ if and only if all its components are zero. Therefore,
\[
\sum_{j=1}^N\left(\sum_{k=0}^M w_k x_j^{k+i} - t_jx_j^i\right) + \lambda w_i = 0, \forall i = 0, \ldots, M
\]
Using the definitions of $A_{ik}$ and $T_i$ introduced in the previous problem, the condition
for extremum becimes
\[
\sum_{k=0}^M A_{ik}w_k = T_i - \lambda w_i.
\]

\item We are given that $p(r) = 0.2, p(b) = 0.2, p(b) = 0.6$. The conditional probabilities are
$p(a|r) = 0.3, p(o|r) = 0.4, p(l|r) = 0.3$, $p(a|b) = 0.5, p(o|b) = 0.5$ and $p(a|g) = 0.3, p(o|g)
= 0.3, p(l|g) = 0.6$.

Now, 
\[
p(a) = p(a|r)p(r) + p(a|b)p(b) + p(a|g)p(g) = 0.34.
\]

If the selected fruit is orange, the probability that it came from the green box is $p(g|o)$. Using
Bayes theorem,
\[
p(g|o) = \frac{p(g, o)}{p(o)} = \frac{p(o|g)p(g)}{p(o)}.
\]
The denominator is calculated as
\[
p(o) = p(o|r)p(r) + p(o|b)p(b) + p(o|g)p(g) = 0.36
\]
Therefore,
\[
p(g|o) = \frac{0.3 \times 0.6}{0.36} = 0.5.
\]

\item Consider equation (1.27) of the book. Instead of writing it with modulus, we write it as
\[
p_y(y) = \pm p_x(g(y))g^\prime(y),
\]
where we chose the sign to make $p_y \ge 0$ for all $y$. The extremum of this function is found
using the relation
\begin{equation}\label{c1pe4}
p_y^\prime(y) = \pm\left(\frac{dp_x}{dg}\left(g^\prime(y)\right)^2 + p_x(g(y))g^{\prime\prime}(y)\right)
\end{equation}
and equating it to zero. If instead of a probability density, we had an ordinary
function $h(y) = f(g(y))$. Then $h^\prime(y) = df/dg g^\prime(y)$ and the condition for extremum
would be $df/dg = 0$ if $g^\prime(y) \ne 0$. The value of $y$ obtained using this relation
is related to the $x$ found using $f^\prime(x) = 0$ is precisely $x = g(y)$. However, it
is not so because of the second term on the right hand side of equation \eqref{c1pe4}.

However, if $g$ is a linear function of $y$ then its second derivative vanishes and the two
equations become similar.

\item We start with the definition $\var(f) = \ev(f(x) - \ev(f(x)))^2 = \ev(f^2(x) - 2f(x)\ev(f(x)) + (\ev(f(x)))^2)
= \ev(f^2(x)) - 2\ev(f(x))\ev(f(x)) + (\ev(f(x))^2) = \ev(f^2(x)) - (\ev(f(x)))^2$, where we 
have used the fact that $\ev(x)$ is a constant and $\ev1(ax) = a\ev(x)$ for any constant $a$.

\item The covariance of two random variables $X$ and $Y$ is given by $\cov(X, Y) = \ev(XY) - \ev(X)\ev(Y)$.
If $p$ is the joint probability density of $X$ and $Y$ and if $p_X$ and $p_Y$ are the respective marginal 
probability densities then,
\begin{eqnarray*}
\ev(XY) &=& \iint p(X, Y)Xy dXdy \\
\ev(X)  &=& \int p_X(X) X dX \\
\ev(Y)  &=& \int p_Y(Y) y dy
\end{eqnarray*}
If $X$ and $Y$ are independent $p(X, Y) = p_{X|Y}(X|Y)p_Y(Y) = p_X(X)p_Y(Y)$ and the $\ev(XY)$ simplifies
to
\[
\ev(XY) = \int p_X(X)XdX \int p_Y(Y)ydy.
\]
The covariance of independent random variables thus vanishes.

\item The proof of normality of the gaussian distribution depends on the integral of $\exp(-ax^2)$
over the real line. Here $a$ is a real constant and $x$ a real variable. Let
\[
I = \int_\sor e^{-ax^2}dx = \int_\sor e^{-ay^2}dy.
\]
Therefore,
\[
I^2 = \iint_{\sor\times\sor}e^{-a(x^2+y^2)}dxdy.
\]
Change the coordinates from $(x, y)$ to $(r, \theta)$, where $x = r\cos\theta$ and $y = r\sin\theta$.
The jacobian of transformation is $r$ so that
\[
I^2 = \int_0^\infty\int_0^{2\pi}e^{-ar^2}rdrd\theta = 2\pi\int_0^\infty e^{-ar^2}rdr.
\]
Introduce the variable $s = r^2$ so that $ds = 2rdr$ and hence,
\[
I^2 = \pi\int_0^\infty e^{-as}ds = \frac{\pi}{a}
\]
so that
\begin{equation}\label{c1pe5}
\int_\sor e^{-ax^2}dx = \sqrt{\frac{\pi}{a}}.
\end{equation}
The normal density is
\[
\mathcal{N}(x|\mu,\sigma) = \frac{1}{\sqrt{2\pi\sigma^2}}\exp\left(-\frac{(x - \mu)^2}{2\sigma^2}\right)
\]
so that
\[
\int_\sor\mathcal{N}(x|\mu,\sigma)dx = \frac{1}{\sqrt{2\pi\sigma^2}}\int_\sor\exp\left(-\frac{(x - \mu)^2}{2\sigma^2}\right)dx.
\]
Introduce the variable $u = (x - \mu)/(\sigma\sqrt{2})$ so that $dx = \sigma\sqrt{2}du$ and the limits of the integral
remain unchanged. Thus,
\[
\int_\sor\mathcal{N}(x|\mu,\sigma)dx = \frac{1}{\sqrt{\pi}}\int_\sor e^{-u^2}du = 1.
\]

\item Let $X$ be a random variable with distribution $\mathcal{N}(x|\mu,\sigma)$. Then its expectation
is
\[
\ev(X) = \int_\sor\mathcal{N}(x|\mu,\sigma)xdx.
\]
Introduce the variable
\begin{equation}\label{c1pe6}
u = \frac{x - \mu}{\sigma\sqrt{2}}
\end{equation}
so that $x = \mu + u\sigma\sqrt{2}$, $dx = \sigma\sqrt{2}du$ and the limits of
the integral remain unchanged. Then,
\[
\ev(X) = \frac{1}{\sqrt{\pi}}\int_\sor e^{-u^2}(\mu + u\sigma\sqrt{2})du = \mu + \sigma\sqrt{\frac{2}{\pi}}\int_\sor ue^{-u^2}du.
\]
The second integral is zero because the integrand is an odd function of $u$ and the limits of the
integral are symmetric around the origin.

\item The variance of a normally distributed random variable is $\var(X) = \ev(X^2) - (\ev X)^2 
= \ev(X^2) - \mu^2$, where we have used the previous exercise. Thus, in order to get the variance
we just need the expectation of $X^2$. It is
\[
\ev(X^2) = \int_\sor\mathcal{N}(x|\mu,\sigma)x^2dx.
\]
We once again change the variable of integration to $u$ defined in equation \eqref{c1pe6} so that
\begin{equation}\label{c1pe7}
\ev(X^2) = \frac{1}{\sqrt{\pi}}\int_\sor e^{-u^2}(\mu^2 + 2\sqrt{2}\mu\sigma u + 2\sigma^2u^2)du.
\end{equation}
The right hand side is the sum of three integrals of which the first one evaluates to $\mu^2$ and
the second one to $0$. The third one is
\begin{equation}\label{c1pe8}
I = \frac{2\sigma^2}{\sqrt{\pi}}\int_\sor e^{-u^2}u^2du.
\end{equation}
In order to evaluate this integral we differentiate equation \eqref{c1pe5} with respect to $a$ to
get
\[
-\int_\sor e^{-ax^2}x^2dx = -\frac{1}{2}\frac{\pi}{a^{3/2}}
\]
or
\begin{equation}\label{c1pe9}
\int_\sor e^{-ax^2}x^2dx = \frac{\sqrt{\pi}}{2}\frac{1}{a^{3/2}}.
\end{equation}
Using equation \eqref{c1pe9} in \eqref{c1pe8} we get $I = \sigma^2$. Therefore, equation \eqref{c1pe7}
becomes
\[
\ev{X^2} = \mu^2 + \sigma^2
\]
from which it immediately follows that $\var{X} = \sigma^2$.


\end{enumerate}
